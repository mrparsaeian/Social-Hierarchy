\documentclass[english, a4paper, 11pt]{article}

\usepackage{microtype}

\usepackage[T1]{fontenc}
\usepackage[utf8]{inputenc}

\usepackage{csquotes}
\usepackage{babel}

\usepackage{amsmath}
\usepackage{amsfonts}
\usepackage{amssymb}

\usepackage[bookmarks=true]{hyperref}
\hypersetup{
	pdftitle={Rat Social adaptive system using  },
	pdfauthor={Mohamad Rasoul Parsaeian},
	pdfkeywords={keyword 1, keyword 2},
	bookmarksnumbered,
	breaklinks=true,
	urlcolor=blue,
	citecolor=black,
	colorlinks=true,
	linkcolor=black,
}

\usepackage{lmodern}
\usepackage{amsmath}
\usepackage{amssymb}
\usepackage{textcomp}

\usepackage[style=ieee, citestyle=ieee]{biblatex}
\bibliography{references}

\usepackage[
	per-mode=symbol,
	%output-decimal-marker={,},
	separate-uncertainty=true,
]{siunitx}

\usepackage{booktabs}
\usepackage{caption}
\captionsetup[table]{skip=1ex}

\usepackage{tikz}
\usepackage{graphicx}
\graphicspath{{figures/}}

\usepackage{pgfgantt}

\usepackage{cleveref}

\usepackage[
    % showframe,
	headheight=16mm,
    bottom=30mm,
]{geometry}

\usepackage{fancyhdr}
\fancypagestyle{unicamp}{
\renewcommand{\headrule}{}
\renewcommand{\footrule}{}
\fancyhead{}
\fancyfoot{}
\fancyhead[L]{\input{unicamp_logo.tex}}
\fancyhead[C]{\sffamily%
{\bfseries\fontsize{15.5pt}{1em}\selectfont\uppercase{Universidade Estadual de Campinas}}\\
\fontsize{11.3pt}{1.2em}\selectfont\uppercase{Faculdade de Engenharia Elétrica e de Computação}\\
\uppercase{Departamento de Comunicações}}
\fancyfoot[C]{\sffamily\fontsize{9pt}{1em}\selectfont%
Av. Albert Einstein, 400; 13083-852 Campinas, SP, Brasil\\
Tel: +55 (19) 3521-3703; Fax: +55 (19) 3289-1395\\
\url{http://www.fee.unicamp.br}}
}

\pagestyle{plain}

\usepackage{setspace}

%\usepackage{parskip}

\usepackage{relsize}
\usepackage[nomain, acronym]{glossaries}
\setacronymstyle{long-sm-short}
\newcommand{\newacronymx}[8][]{%
	\newglossaryentry{#2}{
	type=\acronymtype,
	name={{\smaller #3}},
	sort={#3},
	first={#4 ({\smaller #3}, \emph{#5})},
	firstplural={#7 ({\smaller #6}, \emph{#8})},
	text={{\smaller #3}},
	plural={{\smaller #6}},
	description={#4 (\emph{#5})},#1}}

\newacronym{RTFM}{RTFM}{read the freaking manual}
\newacronymx{AOL}{AOL}{acronym in another language}{acrônimo em outra língua}{AOLs}{acronyms in another language}{acrônimos em outra língua}


\begin{document}

\thispagestyle{unicamp}

\begin{center}

	\null\vfill

	{\scshape\large Research Proposal\par}

	\vskip 3\baselineskip

	{\LARGE\bfseries Title of the Research Proposal\par}

	\vskip 3\baselineskip

	Candidate:\\[1ex]
	{\large\bfseries Candidate Name\par}

	\vskip 3\baselineskip

	Advisor:\\[1ex]
	{\large\bfseries Advisor Name\par}

\end{center}

\vfill

\begin{abstract}
	Abstract text goes here, if needed.
\end{abstract}

\newpage

\onehalfspacing

\section{Introduction}

Main text starts here.
\Cref{fig:example} is an example figure.
There is also \cref{tab:example} as another example.
And don't forget to \gls{RTFM}~\cite{oetiker_not_2015, latex_wikibook}.

\begin{figure}[htp]
	\centering
	\includegraphics[width=3cm]{example}
	\caption{Example figure}
	\label{fig:example}
\end{figure}

\begin{table}[hbp]
	\centering
	\caption{Table description.}
	\label{tab:example}
	\begin{tabular}{lcc}
		\toprule
		Condition     & Frequency (\si{kHz}) & Resistance (\si{\ohm}) \\
		\midrule
		No controller & --                   & \num{0.8}              \\
		Open loop     & \num{120.1}          & \num{45.6}             \\
		Closed loop   & \num{119.3}          & \num{50.1}             \\
		\bottomrule
	\end{tabular}
\end{table}

\section{Objectives}

\section{Methodology}

\section{Schedule of Activities}

The proposed schedule of activities for the project is presented in the Gantt chart in \cref{fig:gantt}.

\begin{figure}[thp]
	\centering
	\begin{ganttchart}[
			hgrid=true,
			vgrid=true,
			canvas/.append style={draw=none},
			title/.append style={draw=none},
			title label font=\small,
			bar label font=\small,
			y unit title=5mm,
			y unit chart=6mm,
			x unit=10mm,
		]{1}{6}
		\gantttitle{2015}{2}
		\gantttitle{2016}{4}\\
		\gantttitlelist{1,...,6}{1}\\
		\ganttbar{1.\ Bibliographical research}{1}{1}\\
		\ganttbar{2.\ Design}{2}{3}\\
		\ganttbar{3.\ Experiments}{3}{6}
	\end{ganttchart}
	\caption{Schedule of activities in trimesters.}
	\label{fig:gantt}
\end{figure}



\section*{Dynamical Systems Model for Social Hierarchy Dynamics}

\subsection*{State Variables}
\begin{align*}
S_i(t) &: \text{Social status of rat } i \text{ at time } t \\
A_i(t) &: \text{Aggressiveness level of rat } i \text{ at time } t \\
R_i(t) &: \text{Resource access level of rat } i \text{ at time } t \\
C_i(t) &: \text{Corticosterone level (stress) of rat } i \text{ at time } t
\end{align*}

\subsection*{Parameters}
\begin{align*}
\alpha &: \text{Influence of social status on resource access} \\
\beta &: \text{Influence of aggressiveness on social status} \\
\gamma &: \text{Influence of stress on aggressiveness} \\
\delta &: \text{Recovery rate of stress} \\
\epsilon &: \text{Environmental factor impact}
\end{align*}

\subsection*{Differential Equations}

\text{Social Status Dynamics:}
\begin{equation}
\frac{dS_i(t)}{dt} = \beta A_i(t) - \epsilon (S_i(t) - \bar{S}(t))
\end{equation}
\text{where } $\bar{S}(t)$ \text{ is the mean social status of the group.}

\text{Aggressiveness Dynamics:}
\begin{equation}
\frac{dA_i(t)}{dt} = \gamma C_i(t) - \epsilon (A_i(t) - \bar{A}(t))
\end{equation}
\text{where } $\bar{A}(t)$ \text{ is the mean aggressiveness of the group.}

\text{Resource Access Dynamics:}
\begin{equation}
\frac{dR_i(t)}{dt} = \alpha S_i(t) - \epsilon (R_i(t) - \bar{R}(t))
\end{equation}
\text{where } $\bar{R}(t)$ \text{ is the mean resource access level of the group.}

\text{Stress Dynamics:}
\begin{equation}
\frac{dC_i(t)}{dt} = \frac{1}{1 + \exp(-\theta (R_i(t) - S_i(t)))} - \delta C_i(t)
\end{equation}
\text{where } $\theta$ \text{ is a parameter determining the sensitivity of stress to the difference between resource access and social status.}

\subsection*{State-Space Representation}
\text{Define the state vector } $\mathbf{x}(t)$ \text{ and input vector } $\mathbf{u}(t)$:
\begin{equation}
\mathbf{x}(t) = \begin{bmatrix} S_i(t) \\ A_i(t) \\ R_i(t) \\ C_i(t) \end{bmatrix}, \quad \mathbf{u}(t) = \epsilon
\end{equation}
\text{The state-space model is:}
\begin{equation}
\frac{d\mathbf{x}(t)}{dt} = \mathbf{A} \mathbf{x}(t) + \mathbf{B} \mathbf{u}(t)
\end{equation}
\text{where:}
\begin{equation}
\mathbf{A} = \begin{bmatrix}
-\epsilon & \beta & 0 & 0 \\
0 & -\epsilon & 0 & \gamma \\
\alpha & 0 & -\epsilon & 0 \\
0 & 0 & \theta (1 - S_i(t)) \frac{\exp(-\theta (R_i(t) - S_i(t)))}{(1 + \exp(-\theta (R_i(t) - S_i(t))))^2} & -\delta
\end{bmatrix}, \quad \mathbf{B} = \begin{bmatrix} 1 \\ 1 \\ 1 \\ 1 \end{bmatrix}
\end{equation}


\section{Conclusion}


\printbibliography

\end{document}
