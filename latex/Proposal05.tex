% ! /////////////////////////////
\documentclass[english, a4paper, 11pt]{article}
\usepackage[english]{babel}
\usepackage{microtype}

\usepackage[T1]{fontenc}
\usepackage[utf8]{inputenc}

\usepackage{csquotes}
\usepackage{babel}
\usepackage{pdflscape}
\usepackage{amsmath}
\usepackage{amsfonts}
\usepackage{amssymb}
\usepackage[hidelinks,colorlinks=true,linkcolor=blue,citecolor=blue]{hyperref}
% \usepackage[bookmarks=true]{hyperref}
% \hypersetup{
% 	pdftitle={Testosterone administration and },
% 	pdfauthor={Mohamad Rasoul Parsaeian},
% 	pdfkeywords={Social Hierarchy, single subject design},
% 	bookmarksnumbered,
% 	breaklinks=true,
% 	urlcolor=blue,
% 	citecolor=black,
% 	colorlinks=true,
% 	linkcolor=black,
% }
\usepackage{listings}
\usepackage{lmodern}
\usepackage{amsmath}
\usepackage{amssymb}
\usepackage{textcomp}

\usepackage[style=apa, citestyle=apa]{biblatex}
% \bibliography{references}
% \bibliography{MyLibrary}
\addbibresource{MyLibrary.bib}
% \bibliographystyle{unsrt}
\usepackage[
	per-mode=symbol,
	%output-decimal-marker={,},
	separate-uncertainty=true,
]{siunitx}

\usepackage{booktabs}
\usepackage{caption}
\captionsetup[table]{skip=1ex}

\usepackage{tikz}
\usetikzlibrary{positioning}

\usepackage{graphicx}
\graphicspath{{figures/}}
\usepackage{pgfgantt}

\usepackage[dvipsnames,svgnames,table]{xcolor}
\ganttset{calendar week text={\small{\startday/\startmonth}}}

\usepackage{cleveref}

\usepackage[
    % showframe,
	headheight=16mm,
    bottom=30mm,
]{geometry}

\usepackage{fancyhdr}
% \fancypagestyle{unicamp}{
% \renewcommand{\headrule}{}
% \renewcommand{\footrule}{}
% \fancyhead{}
% \fancyfoot{}
% \fancyhead[L]{\input{unicamp_logo.tex}}
% \fancyhead[C]{\sffamily%
% {\bfseries\fontsize{15.5pt}{1em}\selectfont\uppercase{IPM - SCC}}\\
% \fontsize{11.3pt}{1.2em}\selectfont\uppercase{Institute for Research}\\
% \fontsize{11.3pt}{1.2em}\selectfont\uppercase{in Fundamental Sciences}\\
% \uppercase{School of Cognitive Science}}
% % \fancyfoot[C]{\sffamily\fontsize{9pt}{1em}\selectfont%
% % Av. Albert Einstein, 400; 13083-852 Campinas, SP, Brasil\\
% % Tel: +55 (19) 3521-3703; Fax: +55 (19) 3289-1395\\
% % \url{http://www.fee.unicamp.br}}
% }

\pagestyle{plain}

\usepackage{setspace}

%\usepackage{parskip}

\usepackage{relsize}
\usepackage[nomain, acronym]{glossaries}
\setacronymstyle{long-sm-short}
\newcommand{\newacronymx}[8][]{%
	\newglossaryentry{#2}{
	type=\acronymtype,
	name={{\smaller #3}},
	sort={#3},
	first={#4 ({\smaller #3}, \emph{#5})},
	firstplural={#7 ({\smaller #6}, \emph{#8})},
	text={{\smaller #3}},
	plural={{\smaller #6}},
	description={#4 (\emph{#5})},#1}}

% \newacronym{RTFM}{RTFM}{read the freaking manual}
% \newacronymx{AOL}{AOL}{acronym in another language}{acrônimo em outra língua}{AOLs}{acronyms in another language}{acrônimos em outra língua}


% ! /////////////////////////////




\title{}
\author{}
\date{}

\begin{document}

\begin{center}

    \null\vfill
    \begin{figure}[htp]
        \centering
        % \includegraphics[width=3cm]{example}
        \includegraphics[width=0.4\textwidth]{figures/IPMLogo.png}
        % \caption{Example figure}
        % \label{fig:example}
    \end{figure}

    {\scshape\large Research Proposal\par}

    \vskip 3\baselineskip

    {\LARGE\bfseries Behavioral and Physiological Consequences of Induced Changes in Social Hierarchies in Male Rats Using the Modified Food Competition Test and Cognitive Modeling via Dynamical Systems Theory: Interplay between Testosterone Administration and Food Access Alterations\par}

    \vskip 3\baselineskip

    By:\\[1ex]
    {\LARGE\bfseries Mohamad Rasoul Parsaeian\par}
    {\small\bfseries m.r.parsa@gmail.com\par}

    \vskip 3\baselineskip

    INSTITUTE FOR RESEARCH
    IN FUNDAMENTAL SCIENCES\\[1ex]
    {\large\bfseries SCHOOL OF COGNITIVE SCIENCE\par}

\end{center}

\vfill

\begin{abstract}
    This research investigates the behavioral and physiological impacts of induced changes in social hierarchies among male rats using an automated system that controls food access through RFID-tagged interactions. Integrating theories such as the frustration-aggression hypothesis (FAH) and dominance hierarchy (DH), we examine how testosterone administration and food access alterations influence social dynamics, aggression, and stress responses within dyads of rats. Control theory principles are employed to design a Social System(SS) where an automated pellet dispenser and testosterone serve as actuators to manipulate or alter social status. Behavioral and physiological data will be collected and analyzed to elucidate the mechanisms underlying social hierarchy formation and maintenance, contributing to the broader understanding of social behavior and its neurobiological foundations.
\end{abstract}

\newpage

\onehalfspacing


% Main text starts here.
% \Cref{fig:example} is an example figure.
% There is also \cref{tab:example} as another example.
% And don't forget to \gls{RTFM}~\cite{oetiker_not_2015, latex_wikibook}.
% And don't forget to \gls{RTFM}.

% \begin{figure}[htp]
%     \centering
%     \includegraphics[width=3cm]{example}
%     \caption{Example figure}
%     \label{fig:example}
% \end{figure}

% \begin{table}[hbp]
%     \centering
%     \caption{Table description.}
%     \label{tab:example}
%     \begin{tabular}{lcc}
%         \toprule
%         Condition     & Frequency (\si{kHz}) & Resistance (\si{\ohm}) \\
%         \midrule
%         No controller & --                   & \num{0.8}              \\
%         Open loop     & \num{120.1}          & \num{45.6}             \\
%         Closed loop   & \num{119.3}          & \num{50.1}             \\
%         \bottomrule
%     \end{tabular}
% \end{table}


\maketitle
% \thispagestyle{unicamp}




\section*{Introduction}
Social hierarchy is a complex trait that significantly influences emotion and cognition in both humans and other social species, affecting social organization, survival, reproductive success, and health within groups. Adapting behavior based on social status can be cost-effective and crucial for survival(\cite{vesseyDominanceControl1981}). Typically, social hierarchies are established through aggressive interactions, serving to manage resources and minimize energy expenditure. Once established, hierarchies reduce aggression by organizing priority access to resources. Behavioral paradigms for measuring social hierarchy in lab animals focus on agonistic interactions over scarce resources or territory defense(\cite{zhouAdvancesUnderstandingNeural2018}).

Social hierarchies in rats can be influenced by various factors, including testosterone levels and access to food resources. Testosterone is known to increase aggression and dominance behaviors, while control over food resources can significantly affect social status. The influence of steroid hormones, especially testosterone in males, on aggression has been extensively studied across various species(\cite{hamiltonSocialNeuroendocrinologyStatus2015}). Experimental manipulation of androgenic signaling has shown that these hormones play a causative role in regulating dominance in both males and females. Testosterone is crucial for activating aggressive behavior in the short-term; it peaks during group formation, rises in dominant individuals after aggressive encounters, and can reinforce winner effects(\cite{fuxjagerSpeciesDifferencesWinner2011,oliveiraWhyWinnersKeep2009}).

\section*{Background}
A seminal research by O.H. Mowrer aimed to extend behavioral psychology to Marx’s economic theories. The "frustration-aggression" hypothesis was applied to Marx and Engels' analysis of class formation in "The Communist Manifesto," suggesting that Marx's materialist history inadvertently mirrored their psychological system(\cite{dollardFrustrationAggression1939}). The Mowrer and his colleagues  reinterpreted Marx’s theory of class conflict as originating from individual frustration with economic confinement. Mowrer not only helped theorize this dynamic but also sought to simulate and film its occurrence. In "An Experimentally Produced 'Social Problem' in Rats" (1939)(\cite{nationallibraryofmedicineExperimentallyProducedSocial2021}) and "Competition and Dominance Hierarchies in Rats" (1940)(\cite{paulofranciscoslompCompetitionDominanceHierarchies2011}), he used film to document social interactions and their psychological effects on individual rats. His films focused on the process of individuation, where hierarchies of behavior emerged in groups of rats through multiple experimental interventions, with each rat developing a distinct identity based on its relationship with the group. These films primarily examined the evolution of group dynamics.


\section*{Research Type}

This is an experimental study designed to investigate the behavioral and physiological consequences of induced changes in social hierarchies among dyads of rats.

\section*{Research Goals}

\begin{itemize}
    \item To determine the dynamics and procedures of social hierarchy formation in rats.
    \item To assess the impact of prioritized food resource access on subordinate rats.
    \item To monitor behavioral and physiological changes in rats due to altered social hierarchies.
\end{itemize}

\section*{Significance}

Understanding the behavioral and physiological impacts of social hierarchy changes in rats can provide insights into social dynamics and stress responses in other social species, including humans. This research could inform interventions for managing social stress and related disorders.

\section*{Variables}

\subsection*{Independent Variable}
\begin{itemize}
    \item Prioritized access to additional food resources (intervention vs. no intervention).
    \item Dosage of testosterone administration for subordinate rate
\end{itemize}

\subsection*{Dependent Variables}
\begin{itemize}
    \item Social Status
    \item Aggressiveness Level
    \item Resource Access Level
    \item Corticosterone Level
    \item Testosterone Level
    \item Food Access Level
\end{itemize}

\section*{Hypotheses}

\begin{itemize}
    \item \textbf{H1}: Subordinate rats given exclusive access to additional food resources will show increased social status over time compared to control rats.
    \item \textbf{H2}: Subordinate rats with additional food access will exhibit lower levels of aggressiveness compared to control rats.
    \item \textbf{H3}: Resource access will be higher in subordinate rats given additional food resources.
    \item \textbf{H4}: Corticosterone levels will be lower in subordinate rats with additional food access due to reduced stress.
    \item \textbf{H5}: Testosterone levels will increase in subordinate rats with additional food access, correlating with improved social status.
    \item \textbf{H6}: Food access levels will be significantly higher in subordinate rats with the intervention.
\end{itemize}

\section*{Experimental Procedure}

\subsection*{Baseline Phase}
\begin{itemize}
    \item Social hierarchy within each dyad of rats is determined using the modified Food Competition test.
    \item The dynamics and procedure of hierarchy formation are recorded.
\end{itemize}

\subsection*{Intervention Phase}
\begin{itemize}
    \item An automated system is activated.
    \item Subordinate rats are given exclusive access to additional food resources through the automated system.
    \item Subordinate rats receive testosterone injections (0.1 mg/kg body weight up to 1 mg/kg body weight every 5 days). .
    \item Interactions between the rats are monitored.
\end{itemize}

\subsection*{Monitoring and Data Collection}
\begin{itemize}
    \item Behavioral observations focus on the frequency and duration of interactions with the sliding door and pellet dispenser.
    \item Physiological measures such as body weight and corticosterone levels are periodically assessed.
\end{itemize}

\section*{Measurement of Variables}

\begin{itemize}
    \item \textbf{Social Status}: Measured by the frequency and outcomes of competitive interactions.
    \item \textbf{Aggressiveness Level}: Assessed using standardized behavioral scoring during interactions.
    \item \textbf{Resource Access Level}: Monitored by the number of times a rat accesses the food dispenser.
    \item \textbf{Corticosterone Level}: Measured through blood samples at regular intervals.
    \item \textbf{Testosterone Level}: Measured through blood samples at regular intervals.
    \item \textbf{Food Access Level}: Tracked by automated system logs of food dispenser use.
\end{itemize}

\section*{Methodology}
\subsection*{Participants}
40 adult male Sprague-Dawley rats, housed in groups of 2 (dyads). Rats will be distributed to dyads consisting of two rats that will be matched for their body weight and anxiety level. The anxiety level will be defined by the time spent in the open arms of the Elevated Plus Maze (EPM)(\cite{herreroIndividualDifferencesAnxiety2006,timmerEvidenceRoleOxytocin2011}).

\subsection*{Interventions}
\begin{itemize}
    \item \textbf{Testosterone Administration}: Subordinate rats will receive testosterone injections (1 mg/kg body weight every 5 days).
    \item \textbf{Food Access Alteration}: Food access will be restricted for certain individuals within each dyad to create competition.
\end{itemize}

\subsection*{Measurements}
\begin{itemize}
    \item \textbf{Social Hierarchy}: Determined using the modified Food Competition test.
    \item \textbf{Behavioral Analysis}: Observations of aggressive and submissive behaviors.
    \item \textbf{Physiological Metrics}: Corticosterone levels, body weight, immune function.
    \item \textbf{Cognitive Function}: Performance in maze tests and problem-solving tasks.
\end{itemize}

\subsection*{Procedure}
\begin{itemize}
    \item Establish baseline hierarchies using the modified Food Competition test.
    \item Apply testosterone administration and food access alterations.
    \item Monitor and record behavioral and physiological responses over 12 weeks.
\end{itemize}

\section*{Modified Food Competition Apparatus}

\begin{figure}[h]
    \centering
    \begin{tikzpicture}
        % Original image
        \node[inner sep=0pt] (image) at (0,0) {\includegraphics[width=0.8\textwidth]{figures/HomeCageModified.png}};
    \end{tikzpicture}
    \caption{Schematic illustration of the home cage in the modified Food Competition apparatus.}
    \label{fig:modified_apparatus}
\end{figure}


\begin{figure}[h]
    \centering
    \begin{tikzpicture}
        % Original image
        \node[inner sep=0pt] (image) at (0,0) {\includegraphics[width=0.8\textwidth]{figures/Feeder.png}};
    \end{tikzpicture}
    \caption{Schematic illustration of the transparent lid and feeder in the modified Food Competition apparatus with modifications.}
    \label{fig:modified_apparatus_feeder}
\end{figure}
\begin{figure}[h]
    \centering
    \begin{tikzpicture}
        % Original image
        \node[inner sep=0pt] (image) at (0,0) {\includegraphics[width=0.8\textwidth]{figures/HomeCageInReality.png}};

    \end{tikzpicture}
    \caption{Schematic illustration of the transparent lid and feeder in the modified Food Competition apparatus with modifications.}
    \label{fig:modified_apparatus_homecage_reality}
\end{figure}

\subsection*{Description of Modifications}

1. \textbf{Automatic Sliding Door}:
\begin{itemize}
    \item A motorized mechanism is installed to automate the sliding door.
    \item An RFID receiver is attached to recognize specific rats and control the door's opening.
\end{itemize}

2. \textbf{Automatic Pellet Release Mechanism}:
\begin{itemize}
    \item The pellet release mechanism is connected to a motorized system.
    \item An RFID receiver is integrated to control pellet release based on the rat's identification.
\end{itemize}
\section*{Procedure}

\subsection*{Automated Sliding Door and Pellet Dispenser System}

To study the behavioral and physiological consequences of induced changes in social hierarchies, we will implement an automated system to control access to food resources based on individual rat identification using RFID tags\cite{habedankMouseWhereArt2020}. This system consists of an automatic sliding door and a pellet dispenser, both responsive to RFID tags to ensure only specific rats can access the resources.

\subsubsection*{RFID Tagging}
\begin{itemize}
    \item Each rat is equipped with a unique RFID tag attached to its collar.
    \item The tags are pre-programmed to correspond to the identity of the subordinate or dominant status of each rat.
\end{itemize}

\subsubsection*{Automated Sliding Door}
\begin{itemize}
    \item The sliding door is equipped with a motorized mechanism controlled by a Arduino micro-controller.
    \item An RFID receiver is installed near the sliding door.
    \item When the RFID tag of a subordinate rat is detected by the receiver, the micro-controller activates the motor to open the door, allowing the rat to access the food area.
    \item If the RFID tag of a dominant rat is detected, the door remains closed.
\end{itemize}

\subsubsection*{Pellet Dispenser}
\begin{itemize}
    \item The pellet dispenser is similarly equipped with a motorized mechanism and an RFID receiver.
    \item Upon detecting the RFID tag of the subordinate rat, the dispenser releases a predetermined number of pellets.
    \item The dispenser remains inactive when the RFID tag of the dominant rat is detected, preventing access to additional food resources.
\end{itemize}

\subsubsection*{Experimental Procedure}
\begin{itemize}
    \item \textbf{Baseline Phase}: Initially, the social hierarchy within each dyads of rats is determined using the modified Food Competition test. The dynamics and procesure of hierarchy formation will be recorded
    \item \textbf{Intervention Phase}: The automated system is activated, and the interactions between the rats are monitored. Subordinate rats are given exclusive access to additional food resources through the automated system.
    \item \textbf{Monitoring and Data Collection}: Behavioral observations are recorded, focusing on the frequency and duration of interactions with the sliding door and pellet dispenser. Physiological measures such as body weight and corticosterone levels are periodically assessed.
\end{itemize}
\section*{Video annotation}
Bonsai(\cite{lopesBonsaiEventbasedFramework2015}) and Python Video Annotator(\cite{ribeiroPythonvideoannotator}), both open-source computer vision software available online, will be used for behavioral quantification. Initially, Bonsai digitally assigned behaviors and created timestamps for each event's start and end. These timestamps will be then curated frame-by-frame using Python Video Annotator, which allows precise subsecond resolution adjustments. Additionally, Python Video Annotator will enable post hoc categorization of exploration behaviors (anticipatory or resource-present) and pushing behaviors (successful or unsuccessful), which can only be identified after the pushing bouts are completed, making online video analysis insufficient for this purpose.
\subsubsection*{Data Analysis}
\begin{itemize}
    \item The data collected from the automated system is analyzed to determine changes in social hierarchy dynamics, food access patterns, and physiological responses.
    \item Statistical analyses are performed to compare the behavior and physiological measures between the subordinate and dominant rats.
\end{itemize}

By implementing this automated system, we ensure precise control over food resource allocation, allowing us to investigate the effects of altered food access on social hierarchy and related behavioral and physiological outcomes.



\section*{Computational Cognitive Model}
\subsection*{State Variables}
\begin{itemize}
    \item \( S_i(t) \): Social status of rat \( i \) at time \( t \)
    \item \( A_i(t) \): Aggressiveness level of rat \( i \) at time \( t \)
    \item \( R_i(t) \): Resource access level of rat \( i \) at time \( t \)
    \item \( C_i(t) \): Corticosterone level of rat \( i \) at time \( t \)
    \item \( T_i(t) \): Testosterone level of rat \( i \) at time \( t \)
    \item \( F_i(t) \): Food access level of rat \( i \) at time \( t \)
\end{itemize}

\subsection*{Differential Equations}
\begin{align*}
    \frac{dS_i(t)}{dt} & = \beta A_i(t) + \xi E_i(t) - \eta D_i(t) - \epsilon (S_i(t) - \bar{S}(t)) + \phi T_i(t) + \lambda F_i(t)  \\
    \frac{dA_i(t)}{dt} & = \gamma C_i(t) + \phi T_i(t) - \epsilon (A_i(t) - \bar{A}(t)) + D \frac{\partial^2 A_i(t)}{\partial x^2}  \\
    \frac{dR_i(t)}{dt} & = \alpha S_i(t) - \epsilon (R_i(t) - \bar{R}(t)) + D \frac{\partial^2 R_i(t)}{\partial x^2}                \\
    \frac{dC_i(t)}{dt} & = \frac{1}{1 + \exp(-\theta (R_i(t) - S_i(t)))} - \delta C_i(t) + D \frac{\partial^2 C_i(t)}{\partial x^2} \\
    \frac{dT_i(t)}{dt} & = \text{Testosterone injection rate} - \delta T_i(t) + D \frac{\partial^2 T_i(t)}{\partial x^2}            \\
    \frac{dF_i(t)}{dt} & = \text{Rate of food access} - \delta F_i(t) + D \frac{\partial^2 F_i(t)}{\partial x^2}
\end{align*}

\section*{Expected Outcomes}
\begin{enumerate}
    \item Testosterone administration and food access alterations will lead to significant shifts in social hierarchy.
    \item Combined interventions will result in increased aggression, altered stress responses, and changes in cognitive performance.
    \item The dynamical systems model will accurately predict the effects of these interventions on social hierarchy dynamics.
\end{enumerate}

\section*{Significance}
This study will provide insights into the mechanisms through which hormonal and environmental factors influence social hierarchies and behavior. The findings will contribute to the understanding of social stress and hierarchy formation in animals, with potential implications for human social dynamics.
%  ! #############################



\section{Schedule of Activities}

The proposed schedule of activities for the project is presented in the Gantt chart in \cref{fig:gantt}.

% \begin{figure}[thp]
%     \centering
%     \begin{ganttchart}[
%             hgrid=true,
%             vgrid=true,
%             canvas/.append style={draw=none},
%             title/.append style={draw=none},
%             title label font=\small,
%             bar label font=\small,
%             y unit title=5mm,
%             y unit chart=6mm,
%             x unit=10mm,
%         ]{1}{8}
%         \gantttitle{2024}{4}
%         \gantttitle{2025}{4}\\
%         \gantttitlelist{1,...,8}{1}\\
%         \ganttbar{1.\ Literature Review}{1}{1}\\
%         \ganttbar{2.\ Tuning Design}{2}{3}\\
%         \ganttbar{3.\ Developing the Home Cage, SOftware and hardware}{2}{3}\\
%         \ganttbar{4.\ Experiment and Data Gathering}{3}{6}\\
%         \ganttbar{5.\ Data Analysing}{5}{8}\\
%         \ganttbar{6.\ Writing Paper}{6}{8}\\
%         \ganttbar{7.\ Submitting Paper}{7}{8}
%     \end{ganttchart}
%     \caption{Schedule of activities in trimesters.}
%     \label{fig:gantt}
% \end{figure}
\newpage

\begin{landscape}
    \begin{figure}[h!bt]
        \begin{center}
            \begin{ganttchart}[
                    hgrid,
                    vgrid={*6{draw=none}, dotted},
                    bar/.append style={fill=black},
                    bar incomplete/.append style={fill=white},
                    time slot format=isodate,
                    time slot format/base century=2000,
                    x unit=0.062cm,
                    y unit chart=0.6cm,
                    y unit title=0.6cm, % height of title line and gap
                    title height=1, % use full height for title, leaving no gap
                    bar top shift=0.1,
                    bar height=0.8,
                    title label font=\bfseries\normalsize,
                    time slot format/start date=2024-10-01]{2024-10-01}{2025-09-28}
                % time slot format/start date=2018-01-01]{2018-01-01}{2018-12-30}
                \gantttitle{Schedule of Activities}{363}\\
                \gantttitlecalendar{year, month=shortname}\\
                %                   increase height   rotate label
                \gantttitlecalendar[title height=1.8, title label node/.append style={rotate=90}]{week}\\
                \gantttitle[title/.style={opacity=0}]{}{364}\\ % invisible title to make room for previous higher line
                % \gantttitle[title/.style={opacity=0}]{}{364}\\ % invisible title to make room for previous higher line
                \ganttbar[bar/.append style={fill=cyan}]{Literature Review}{2024-10-01}{2024-11-15}\\
                \ganttbar[bar/.append style={fill=cyan}]{Tuning Design}{2024-10-01}{2024-12-01}\\
                \ganttbar[bar/.append style={fill=cyan}]{Building Apparatus}{2024-11-02}{2024-12-01}\\
                \ganttbar[bar/.append style={fill=cyan}]{Data Gathering}{2024-12-02}{2025-03-01}\\
                \ganttbar[bar/.append style={fill=cyan}]{Data Analysing}{2025-02-02}{2025-04-01}\\
                \ganttbar[bar/.append style={fill=cyan}]{Draft Paper}{2025-03-15}{2025-05-01}\\
                \ganttbar[bar/.append style={fill=cyan}]{Reviewing Paper}{2025-05-01}{2025-08-01}\\
                \ganttbar[bar/.append style={fill=cyan}]{Submitting Paper}{2025-05-15}{2025-09-28}
            \end{ganttchart}
        \end{center}
        \label{fig:gantt}

    \end{figure}
\end{landscape}
\newpage

% \section*{Dynamical Systems Model for Social Hierarchy Dynamics}

% \subsection*{State Variables}
% \begin{align*}
%     S_i(t) & : \text{Social status of rat } i \text{ at time } t                 \\
%     A_i(t) & : \text{Aggressiveness level of rat } i \text{ at time } t          \\
%     R_i(t) & : \text{Resource access level of rat } i \text{ at time } t         \\
%     C_i(t) & : \text{Corticosterone level (stress) of rat } i \text{ at time } t
% \end{align*}

% \subsection*{Parameters}
% \begin{align*}
%     \alpha   & : \text{Influence of social status on resource access} \\
%     \beta    & : \text{Influence of aggressiveness on social status}  \\
%     \gamma   & : \text{Influence of stress on aggressiveness}         \\
%     \delta   & : \text{Recovery rate of stress}                       \\
%     \epsilon & : \text{Environmental factor impact}
% \end{align*}

% \subsection*{Differential Equations}

% \text{Social Status Dynamics:}
% \begin{equation}
%     \frac{dS_i(t)}{dt} = \beta A_i(t) - \epsilon (S_i(t) - \bar{S}(t))
% \end{equation}
% \text{where } $\bar{S}(t)$ \text{ is the mean social status of the group.}

% \text{Aggressiveness Dynamics:}
% \begin{equation}
%     \frac{dA_i(t)}{dt} = \gamma C_i(t) - \epsilon (A_i(t) - \bar{A}(t))
% \end{equation}
% \text{where } $\bar{A}(t)$ \text{ is the mean aggressiveness of the group.}

% \text{Resource Access Dynamics:}
% \begin{equation}
%     \frac{dR_i(t)}{dt} = \alpha S_i(t) - \epsilon (R_i(t) - \bar{R}(t))
% \end{equation}
% \text{where } $\bar{R}(t)$ \text{ is the mean resource access level of the group.}

% \text{Stress Dynamics:}
% \begin{equation}
%     \frac{dC_i(t)}{dt} = \frac{1}{1 + \exp(-\theta (R_i(t) - S_i(t)))} - \delta C_i(t)
% \end{equation}
% \text{where } $\theta$ \text{ is a parameter determining the sensitivity of stress to the difference between resource access and social status.}

% \subsection*{State-Space Representation}
% \text{Define the state vector } $\mathbf{x}(t)$ \text{ and input vector } $\mathbf{u}(t)$:
% \begin{equation}
%     \mathbf{x}(t) = \begin{bmatrix} S_i(t) \\ A_i(t) \\ R_i(t) \\ C_i(t) \end{bmatrix}, \quad \mathbf{u}(t) = \epsilon
% \end{equation}
% \text{The state-space model is:}
% \begin{equation}
%     \frac{d\mathbf{x}(t)}{dt} = \mathbf{A} \mathbf{x}(t) + \mathbf{B} \mathbf{u}(t)
% \end{equation}
% \text{where:}
% \begin{equation}
%     \mathbf{A} = \begin{bmatrix}
%         -\epsilon & \beta     & 0                                                                                                   & 0       \\
%         0         & -\epsilon & 0                                                                                                   & \gamma  \\
%         \alpha    & 0         & -\epsilon                                                                                           & 0       \\
%         0         & 0         & \theta (1 - S_i(t)) \frac{\exp(-\theta (R_i(t) - S_i(t)))}{(1 + \exp(-\theta (R_i(t) - S_i(t))))^2} & -\delta
%     \end{bmatrix}, \quad \mathbf{B} = \begin{bmatrix} 1 \\ 1 \\ 1 \\ 1 \end{bmatrix}
% \end{equation}

\section*{Data Analysis using Repeated Measures ANOVA}

In this section, we describe the process of analyzing changes over time in the experimental data using Repeated Measures ANOVA. This statistical method is suitable for comparing the means of three or more groups where the same subjects are used in each group, making it ideal for analyzing longitudinal data from the same group of rats across different time points.

\subsection*{Steps for Data Analysis}

\begin{enumerate}
    \item \textbf{Data Preparation}:
          \begin{itemize}
              \item Ensure the data is structured appropriately, with each row representing a time point and each column representing a measurement for a specific rat.
              \item Organize the data into a format suitable for repeated measures analysis, typically with columns for subject identifiers, time points, and the variable of interest.
          \end{itemize}
    \item \textbf{Statistical Assumptions}:
          \begin{itemize}
              \item Check for sphericity, which is the assumption that the variances of the differences between all combinations of related groups are equal. If sphericity is violated, adjustments such as the Greenhouse-Geisser correction may be applied.
          \end{itemize}
    \item \textbf{Performing Repeated Measures ANOVA}:
          \begin{itemize}
              \item Conduct the ANOVA to test for significant differences over time within subjects.
          \end{itemize}
    \item \textbf{Post-Hoc Analysis}:
          \begin{itemize}
              \item If the ANOVA indicates significant differences, perform post-hoc tests to identify which time points differ from each other.
          \end{itemize}
\end{enumerate}

\subsection*{Python Code for generation test dataset and Repeated Measures ANOVA}

\lstset{language=Python, basicstyle=\ttfamily\footnotesize, keywordstyle=\color{blue}, commentstyle=\color{green}, stringstyle=\color{red}, breaklines=true, frame=single}
\begin{lstlisting}
    import numpy as np
    import pandas as pd
    from scipy.integrate import solve_ivp
    from statsmodels.stats.anova import AnovaRM
    from statsmodels.stats.multicomp import pairwise_tukeyhsd
    import matplotlib.pyplot as plt
    
    # Simulation parameters
    num_rats = 10
    t_points = 50
    np.random.seed(42)
    
    # Simulate time series data
    def simulate_data(num_rats, t_points):
        time = np.arange(t_points)
        data = {
            'Time': np.tile(time, num_rats),
            'Rat': np.repeat(np.arange(num_rats), t_points),
            'Social_Status': np.random.rand(num_rats * t_points),
            'Aggressiveness': np.random.rand(num_rats * t_points),
            'Resource_Access': np.random.rand(num_rats * t_points),
            'Corticosterone': np.random.rand(num_rats * t_points),
            'Testosterone': np.random.rand(num_rats * t_points),
            'Food_Access': np.random.rand(num_rats * t_points)
        }
        return pd.DataFrame(data)
    
    # Generate the data
    df = simulate_data(num_rats, t_points)
    
    # Perform Repeated Measures ANOVA for each attribute
    attributes = ['Social_Status', 'Aggressiveness', 'Resource_Access', 'Corticosterone', 'Testosterone', 'Food_Access']
    
    for attribute in attributes:
        print(f'\nRepeated Measures ANOVA for {attribute}')
        aovrm = AnovaRM(df, attribute, 'Rat', within=['Time'])
        res = aovrm.fit()
        print(res)
    
        # Example of post-hoc analysis (if needed)
        posthoc = pairwise_tukeyhsd(df[attribute], df['Time'])
        print(posthoc)
    
    # Visualizing the simulated data (optional)
    for attribute in attributes:
        plt.figure(figsize=(10, 6))
        for rat in range(num_rats):
            plt.plot(df['Time'][df['Rat'] == rat], df[attribute][df['Rat'] == rat], label=f'Rat {rat}')
        plt.title(f'{attribute} Over Time')
        plt.xlabel('Time')
        plt.ylabel(attribute)
        plt.legend(loc='upper right')
        plt.show()
    
\end{lstlisting}

\section*{Model Selection and Fitting}

Model selection and fitting are crucial steps in accurately capturing the dynamics of social hierarchies in rats based on the experimental data. This section describes the process of selecting the appropriate model, fitting the model to the data, and estimating the parameters using Python.

\subsection*{Model Selection}

Model selection involves comparing different models to determine which best describes the observed data. For this experiment, we consider models based on differential equations, such as linear regression, logistic regression, and more complex dynamical systems models.

\subsubsection*{Steps for Model Selection}

\begin{enumerate}
    \item \textbf{Define Candidate Models}
          \begin{itemize}
              \item Linear regression
              \item Logistic regression
              \item Dynamical systems model (e.g., using ordinary differential equations)
          \end{itemize}
    \item \textbf{Model Evaluation Metrics}
          \begin{itemize}
              \item Akaike Information Criterion (AIC)
              \item Bayesian Information Criterion (BIC)
              \item Mean Squared Error (MSE)
              \item R-squared value for goodness-of-fit
          \end{itemize}
    \item \textbf{Cross-Validation}
          \begin{itemize}
              \item Perform k-fold cross-validation to assess model performance on different subsets of data.
          \end{itemize}
\end{enumerate}

\subsection*{Model Fitting and Parameter Estimation}

Once the best model is selected, the next step is fitting the model to the data and estimating the parameters. This involves using optimization techniques to minimize the error between the predicted and observed values.


\subsubsection*{Steps for Model Fitting and Parameter Estimation}

\begin{enumerate}
    \item \textbf{Define the Objective Function}
          \begin{itemize}
              \item The objective function could be the mean squared error between predicted and observed values.
          \end{itemize}
    \item \textbf{Optimization Algorithm}
          \begin{itemize}
              \item Use optimization algorithms such as gradient descent, Newton's method, or more advanced methods like the Levenberg-Marquardt algorithm for non-linear models.
          \end{itemize}
    \item \textbf{Parameter Estimation}
          \begin{itemize}
              \item Estimate the parameters that minimize the objective function.
          \end{itemize}
\end{enumerate}

\subsection*{Python Code for Simulation, Model Selection, Model Fitting, and Parameter Estimation}

\lstset{language=Python, basicstyle=\ttfamily\footnotesize, keywordstyle=\color{blue}, commentstyle=\color{green}, stringstyle=\color{red}, breaklines=true, frame=single}
\begin{lstlisting}
    import numpy as np
    import pandas as pd
    from scipy.integrate import solve_ivp
    from scipy.optimize import minimize
    from sklearn.model_selection import KFold
    from sklearn.metrics import mean_squared_error
    from sklearn.linear_model import LinearRegression
    import matplotlib.pyplot as plt
    
    # Define the testosterone injection rate function
    def testosterone_injection_rate(t, period=5):
        return 1.0 if (t // period) % 2 == 0 else 0.0
    
    # Define the defeat stress exposure rate function
    def defeat_stress_exposure_rate(t, period=5):
        return 1.0 if (t // period) % 2 == 0 else 0.0
    
    # Define the dynamical systems model using ODEs
    def ode_system(t, state, alpha, beta, gamma, delta, epsilon, theta, phi, xi, eta):
        S, A, R, C, T, E, D = state[:7]
        dSdt = beta * A + xi * E - eta * D - epsilon * (S - np.mean(S))
        dAdt = gamma * C + phi * T - epsilon * (A - np.mean(A))
        dRdt = alpha * S - epsilon * (R - np.mean(R))
        dCdt = 1 / (1 + np.exp(-theta * (R - S))) - delta * C
        dTdt = testosterone_injection_rate(t) - delta * T
        dEdt = 0.05 - delta * E
        dDdt = defeat_stress_exposure_rate(t) - delta * D
        return [dSdt, dAdt, dRdt, dCdt, dTdt, dEdt, dDdt]
    
    # Example function to generate synthetic data
    def generate_data(params, num_rats=50, t_points=100):
        initial_conditions = [0.5] * 7
        t = np.linspace(0, 50, t_points)
        sol = solve_ivp(ode_system, [t[0], t[-1]], initial_conditions, t_eval=t, args=params, method='RK45')
        S, A, R, C, T, E, D = sol.y
        return t, S, A, R, C, T, E, D
    
    # Example function for model fitting using optimization
    def fit_model(data, initial_conditions, params_guess):
        def objective_function(params):
            t, S_obs, A_obs, R_obs, C_obs, T_obs, E_obs, D_obs = data
            sol = solve_ivp(ode_system, [t[0], t[-1]], initial_conditions, t_eval=t, args=params, method='RK45')
            S_pred, A_pred, R_pred, C_pred, T_pred, E_pred, D_pred = sol.y
            mse = mean_squared_error(S_obs, S_pred) # Update this line to ensure correct shape
            return mse
    
        result = minimize(objective_function, params_guess, method='L-BFGS-B')
        return result.x
    
    # Example function for model selection using cross-validation
    def model_selection(data, models, scoring):
        kf = KFold(n_splits=5)
        best_model = None
        best_score = float('inf')
    
        for model in models:
            scores = []
            for train_index, test_index in kf.split(data):
                train_data, test_data = data[train_index], data[test_index]
                model.fit(train_data[:, :-1], train_data[:, -1])
                predictions = model.predict(test_data[:, :-1])
                score = scoring(test_data[:, -1], predictions)
                scores.append(score)
    
            avg_score = np.mean(scores)
            if avg_score < best_score:
                best_score = avg_score
                best_model = model
    
        return best_model
    
    # Define candidate models
    models = [
        LinearRegression()
    ]
    
    # Example experimental data
    params = (1.0, 0.5, 0.1, 0.1, 0.05, 2.0, 0.3, 0.2, 0.4)
    t, S, A, R, C, T, E, D = generate_data(params=params, num_rats=50, t_points=100)
    
    # Define initial conditions
    initial_conditions = [0.5] * 7
    
    # Flatten the data for model selection
    data = np.hstack([S[:, np.newaxis], A[:, np.newaxis], R[:, np.newaxis],
                      C[:, np.newaxis], T[:, np.newaxis], E[:, np.newaxis], 
                      D[:, np.newaxis]])
    
    # Model selection using mean squared error
    best_model = model_selection(data, models, scoring=mean_squared_error)
    
    # Fit the selected model to the data
    params_guess = [1.0, 0.5, 0.1, 0.1, 0.05, 2.0, 0.3, 0.2, 0.4]
    best_params = fit_model((t, S, A, R, C, T, E, D), initial_conditions, params_guess)
    
    print("Best Model Parameters:", best_params)
    
\end{lstlisting}


\printbibliography
\end{document}
