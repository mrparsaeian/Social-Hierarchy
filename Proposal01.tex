% \usepackage{amsmath}
% \usepackage{graphicx}
% \usepackage{tikz}
\usetikzlibrary{positioning}


% ! /////////////////////////////
\documentclass[english, a4paper, 11pt]{article}

\usepackage{microtype}

\usepackage[T1]{fontenc}
\usepackage[utf8]{inputenc}

\usepackage{csquotes}
\usepackage{babel}

\usepackage{amsmath}
\usepackage{amsfonts}
\usepackage{amssymb}

\usepackage[bookmarks=true]{hyperref}
\hypersetup{
	pdftitle={Testosterone administration and },
	pdfauthor={Mohamad Rasoul Parsaeian},
	pdfkeywords={Social Hierarchy, single subject design},
	bookmarksnumbered,
	breaklinks=true,
	urlcolor=blue,
	citecolor=black,
	colorlinks=true,
	linkcolor=black,
}

\usepackage{lmodern}
\usepackage{amsmath}
\usepackage{amssymb}
\usepackage{textcomp}

\usepackage[style=ieee, citestyle=ieee]{biblatex}
\bibliography{references}

\usepackage[
	per-mode=symbol,
	%output-decimal-marker={,},
	separate-uncertainty=true,
]{siunitx}

\usepackage{booktabs}
\usepackage{caption}
\captionsetup[table]{skip=1ex}

\usepackage{tikz}
\usepackage{graphicx}
\graphicspath{{figures/}}

\usepackage{pgfgantt}

\usepackage{cleveref}

\usepackage[
    % showframe,
	headheight=16mm,
    bottom=30mm,
]{geometry}

\usepackage{fancyhdr}
\fancypagestyle{unicamp}{
\renewcommand{\headrule}{}
\renewcommand{\footrule}{}
\fancyhead{}
\fancyfoot{}
\fancyhead[L]{\input{unicamp_logo.tex}}
\fancyhead[C]{\sffamily%
{\bfseries\fontsize{15.5pt}{1em}\selectfont\uppercase{Universidade Estadual de Campinas}}\\
\fontsize{11.3pt}{1.2em}\selectfont\uppercase{Faculdade de Engenharia Elétrica e de Computação}\\
\uppercase{Departamento de Comunicações}}
\fancyfoot[C]{\sffamily\fontsize{9pt}{1em}\selectfont%
Av. Albert Einstein, 400; 13083-852 Campinas, SP, Brasil\\
Tel: +55 (19) 3521-3703; Fax: +55 (19) 3289-1395\\
\url{http://www.fee.unicamp.br}}
}

\pagestyle{plain}

\usepackage{setspace}

%\usepackage{parskip}

\usepackage{relsize}
\usepackage[nomain, acronym]{glossaries}
\setacronymstyle{long-sm-short}
\newcommand{\newacronymx}[8][]{%
	\newglossaryentry{#2}{
	type=\acronymtype,
	name={{\smaller #3}},
	sort={#3},
	first={#4 ({\smaller #3}, \emph{#5})},
	firstplural={#7 ({\smaller #6}, \emph{#8})},
	text={{\smaller #3}},
	plural={{\smaller #6}},
	description={#4 (\emph{#5})},#1}}

\newacronym{RTFM}{RTFM}{read the freaking manual}
\newacronymx{AOL}{AOL}{acronym in another language}{acrônimo em outra língua}{AOLs}{acronyms in another language}{acrônimos em outra língua}


% ! /////////////////////////////




\title{Behavioral and Physiological Consequences of Induced Changes in Social Hierarchies in Male Rats Using the Modified Food Competition Test and Cognitive Modeling via Dynamical Systems Theory: Interplay between Testosterone Administration and Food Access Alterations}
\author{Mohamad Rasoul Parsaeian}
\date{}

\begin{document}

\maketitle

\section*{Background}
Social hierarchies in rats can be influenced by various factors, including testosterone levels and access to food resources. Testosterone is known to increase aggression and dominance behaviors, while control over food resources can significantly affect social status.

\section*{Objectives}
\begin{enumerate}
	\item To investigate how the administration of testosterone and alterations in food access together affect the social hierarchy in triads of male rats.
	\item To assess the combined behavioral and physiological consequences of these interventions.
	\item To develop a dynamical systems model incorporating these factors to predict changes in social hierarchy.
\end{enumerate}

\section*{Hypotheses}
\begin{enumerate}
	\item Administration of testosterone will increase dominance behaviors and alter social hierarchies.
	\item Restricting food access for certain individuals will lead to increased competition and shifts in social status.
	\item The combination of testosterone administration and food access alteration will have a synergistic effect on social hierarchy dynamics.
\end{enumerate}

\section*{Methodology}
\subsection*{Participants}
40 adult male Sprague-Dawley rats, housed in groups of 3 (triads).

\subsection*{Interventions}
\begin{itemize}
	\item \textbf{Testosterone Administration}: Subordinate rats will receive testosterone injections (1 mg/kg body weight every 5 days).
	\item \textbf{Food Access Alteration}: Food access will be restricted for certain individuals within each triad to create competition.
\end{itemize}

\subsection*{Measurements}
\begin{itemize}
	\item \textbf{Social Hierarchy}: Determined using the modified Food Competition test.
	\item \textbf{Behavioral Analysis}: Observations of aggressive and submissive behaviors.
	\item \textbf{Physiological Metrics}: Corticosterone levels, body weight, immune function.
	\item \textbf{Cognitive Function}: Performance in maze tests and problem-solving tasks.
\end{itemize}

\subsection*{Procedure}
\begin{itemize}
	\item Establish baseline hierarchies using the modified Food Competition test.
	\item Apply testosterone administration and food access alterations.
	\item Monitor and record behavioral and physiological responses over 12 weeks.
\end{itemize}

\section*{Modified Food Competition Apparatus}

\begin{figure}[h]
    \centering
    \begin{tikzpicture}
        % Original image
        \node[inner sep=0pt] (image) at (0,0) {\includegraphics[width=0.8\textwidth]{figures/HomeCage.png}};
        
        % % Automatic Sliding Door
        % \draw[red, thick] (-3.5,-1) rectangle (-2,-2.5);
        % \node[red, below=of image, yshift=1.5cm, xshift=-2.5cm] {Automatic Sliding Door};

        % % RFID Receiver for Sliding Door
        % \draw[blue, thick] (-2,-1) circle (0.5);
        % \node[blue, below=of image, yshift=1.8cm, xshift=-1.8cm] {RFID Receiver};

        % % Automatic Pellet Release Mechanism
        % \draw[green, thick] (-6,-2) rectangle (-4.5,-3.5);
        % \node[green, below=of image, yshift=3.5cm, xshift=-5.5cm] {Automatic Pellet Release};

        % % RFID Receiver for Pellet Release
        % \draw[orange, thick] (-4.5,-2) circle (0.5);
        % \node[orange, below=of image, yshift=3.8cm, xshift=-4.3cm] {RFID Receiver};
    \end{tikzpicture}
    \caption{Schematic illustration of the home cage in the modified Food Competition apparatus.}
    \label{fig:modified_apparatus}
\end{figure}


\begin{figure}[h]
    \centering
    \begin{tikzpicture}
        % Original image
        \node[inner sep=0pt] (image) at (0,0) {\includegraphics[width=0.8\textwidth]{figures/Feeder.png}};
        
        % % Automatic Sliding Door
        % \draw[red, thick] (-3.5,-1) rectangle (-2,-2.5);
        % \node[red, below=of image, yshift=1.5cm, xshift=-2.5cm] {Automatic Sliding Door};

        % % RFID Receiver for Sliding Door
        % \draw[blue, thick] (-2,-1) circle (0.5);
        % \node[blue, below=of image, yshift=1.8cm, xshift=-1.8cm] {RFID Receiver};

        % % Automatic Pellet Release Mechanism
        % \draw[green, thick] (-6,-2) rectangle (-4.5,-3.5);
        % \node[green, below=of image, yshift=3.5cm, xshift=-5.5cm] {Automatic Pellet Release};

        % % RFID Receiver for Pellet Release
        % \draw[orange, thick] (-4.5,-2) circle (0.5);
        % \node[orange, below=of image, yshift=3.8cm, xshift=-4.3cm] {RFID Receiver};
    \end{tikzpicture}
    \caption{Schematic illustration of the transparent lid and feeder in the modified Food Competition apparatus with modifications.}
    \label{fig:modified_apparatus_feeder}
\end{figure}
\begin{figure}[h]
    \centering
    \begin{tikzpicture}
        % Original image
        \node[inner sep=0pt] (image) at (0,0) {\includegraphics[width=0.8\textwidth]{figures/HomeCageInReality.png}};
        
        % % Automatic Sliding Door
        % \draw[red, thick] (-3.5,-1) rectangle (-2,-2.5);
        % \node[red, below=of image, yshift=1.5cm, xshift=-2.5cm] {Automatic Sliding Door};

        % % RFID Receiver for Sliding Door
        % \draw[blue, thick] (-2,-1) circle (0.5);
        % \node[blue, below=of image, yshift=1.8cm, xshift=-1.8cm] {RFID Receiver};

        % % Automatic Pellet Release Mechanism
        % \draw[green, thick] (-6,-2) rectangle (-4.5,-3.5);
        % \node[green, below=of image, yshift=3.5cm, xshift=-5.5cm] {Automatic Pellet Release};

        % % RFID Receiver for Pellet Release
        % \draw[orange, thick] (-4.5,-2) circle (0.5);
        % \node[orange, below=of image, yshift=3.8cm, xshift=-4.3cm] {RFID Receiver};
    \end{tikzpicture}
    \caption{Schematic illustration of the transparent lid and feeder in the modified Food Competition apparatus with modifications.}
    \label{fig:modified_apparatus_homecage_reality}
\end{figure}

\subsection*{Description of Modifications}

1. \textbf{Automatic Sliding Door}:
   \begin{itemize}
       \item A motorized mechanism is installed to automate the sliding door.
       \item An RFID receiver is attached to recognize specific rats and control the door's opening.
   \end{itemize}

2. \textbf{Automatic Pellet Release Mechanism}:
   \begin{itemize}
       \item The pellet release mechanism is connected to a motorized system.
       \item An RFID receiver is integrated to control pellet release based on the rat's identification.
   \end{itemize}

\section*{Computational Cognitive Model}
\subsection*{State Variables}
\begin{itemize}
	\item \( S_i(t) \): Social status of rat \( i \) at time \( t \)
	\item \( A_i(t) \): Aggressiveness level of rat \( i \) at time \( t \)
	\item \( R_i(t) \): Resource access level of rat \( i \) at time \( t \)
	\item \( C_i(t) \): Corticosterone level of rat \( i \) at time \( t \)
	\item \( T_i(t) \): Testosterone level of rat \( i \) at time \( t \)
	\item \( F_i(t) \): Food access level of rat \( i \) at time \( t \)
\end{itemize}

\subsection*{Differential Equations}
\begin{align*}
	\frac{dS_i(t)}{dt} & = \beta A_i(t) + \xi E_i(t) - \eta D_i(t) - \epsilon (S_i(t) - \bar{S}(t)) + \phi T_i(t) + \lambda F_i(t)  \\
	\frac{dA_i(t)}{dt} & = \gamma C_i(t) + \phi T_i(t) - \epsilon (A_i(t) - \bar{A}(t)) + D \frac{\partial^2 A_i(t)}{\partial x^2}  \\
	\frac{dR_i(t)}{dt} & = \alpha S_i(t) - \epsilon (R_i(t) - \bar{R}(t)) + D \frac{\partial^2 R_i(t)}{\partial x^2}                \\
	\frac{dC_i(t)}{dt} & = \frac{1}{1 + \exp(-\theta (R_i(t) - S_i(t)))} - \delta C_i(t) + D \frac{\partial^2 C_i(t)}{\partial x^2} \\
	\frac{dT_i(t)}{dt} & = \text{Testosterone injection rate} - \delta T_i(t) + D \frac{\partial^2 T_i(t)}{\partial x^2}            \\
	\frac{dF_i(t)}{dt} & = \text{Rate of food access} - \delta F_i(t) + D \frac{\partial^2 F_i(t)}{\partial x^2}
\end{align*}

\section*{Expected Outcomes}
\begin{enumerate}
	\item Testosterone administration and food access alterations will lead to significant shifts in social hierarchy.
	\item Combined interventions will result in increased aggression, altered stress responses, and changes in cognitive performance.
	\item The dynamical systems model will accurately predict the effects of these interventions on social hierarchy dynamics.
\end{enumerate}

\section*{Significance}
This study will provide insights into the mechanisms through which hormonal and environmental factors influence social hierarchies and behavior. The findings will contribute to the understanding of social stress and hierarchy formation in animals, with potential implications for human social dynamics.

\end{document}
